\documentclass[12pt,fleqn]{article}
\date{2015-01-09+}

\usepackage{
    amsfonts,
    amsmath,
    amssymb,
    amsthm,
    bashful,
    booktabs,
    enumerate,
    flexisym,
    graphicx,
    setspace,
    slantsc,
    tabularx,
    url,
}
\urlstyle{same}
\newcolumntype{C}{>{\centering\arraybackslash}X}

\usepackage[small]{caption}
\usepackage[T1]{fontenc}
\usepackage[margin=1.25in]{geometry}
\usepackage[charter]{mathdesign}
\usepackage[letterspace=35]{microtype}
\usepackage[sort,round,comma]{natbib}
\bibliographystyle{abbrvnat}
\newcommand\citepos[2][]{\citeauthor{#2}'s \citeyearpar[#1]{#2}}
\newcommand\poscw{\citeauthor{ClW:06}'s \citeyearpar{ClW:06,ClW:07}}
\newcommand\citen[1]{\citeauthor{#1}, \citeyear{#1}}

\usepackage{fancyhdr}
\pagestyle{fancy}
\renewcommand{\sectionmark}[1]{\markboth{}{\footnotesize{\thesection. #1}}}
\renewcommand{\subsectionmark}[1]{\markboth{}{\footnotesize{\thesubsection. #1}}}
\renewcommand{\headrulewidth}{0pt}
\renewcommand{\footrulewidth}{0pt}
\fancyhead[lo,le]{\textit{\footnotesize{\nouppercase{\rightmark}}}}
\fancyhead[ro,re]{\textit{\footnotesize{\VERSION}}}

\newtheorem{innerrthm}{Theorem}
\newenvironment{rthm}[1]
  {\renewcommand\theinnerrthm{#1}\innerrthm}
  {\endinnerrthm}

\newtheorem{innerrlem}{Lemma}
\newenvironment{rlem}[1]
  {\renewcommand\theinnerrlem{#1}\innerrlem}
  {\endinnerrlem}

\newtheorem{innercustomasmp}{Assumption}
\newenvironment{customasmp}[1]
  {\renewcommand\theinnercustomasmp{#1}\innercustomasmp}
  {\endinnercustomasmp}

\newtheorem{thm}{Theorem}
\newtheorem{lem}[thm]{Lemma}
\newtheorem{claim}[thm]{Claim}
\newtheorem{cor}[thm]{Corollary}
\newtheorem{athm}{Theorem}[section]
\newtheorem{alem}[athm]{Lemma}
\newtheorem{alg}{Algorithm}
\newtheorem{asmp}{Assumption}

\theoremstyle{definition}
\newtheorem{defn}{Definition}
\newtheorem{rem}{Remark}

\DeclareMathOperator*{\argmin}{arg\,min}
\DeclareMathOperator{\cov}{cov}
\DeclareMathOperator{\corr}{corr}
\DeclareMathOperator{\diag}{diag}
\DeclareMathOperator{\E}{E}
\DeclareMathOperator{\var}{var}
\DeclareMathOperator*{\plim}{plim}
%\DeclareMathOperator{\vec}{vec}
\DeclareMathOperator{\vech}{vech}

\DeclareMathOperator{\pr}{Pr}

\newcommand{\iid}{\ensuremath\mathit{i.i.d.}}
\newcommand{\bernoulli}{\ensuremath\mathit{bernoulli}}

\newcommand{\Bh}{\hat{B}}
\newcommand{\btrue}[1][]{\if#1*\hat\beta_{T+1}\else\beta_0\fi}
\newcommand{\bh}{\hat{\beta}}
\newcommand{\bt}{\tilde{\beta}}
\newcommand{\ep}{\varepsilon}
\newcommand{\eph}{\hat{\varepsilon}}
\newcommand{\Fb}{\bar{F}}
\newcommand{\Fh}{\hat{F}}
\newcommand{\Fs}{\mathcal{F}}
\newcommand{\fb}{\bar{f}}
\newcommand{\fh}{\hat{f}}
\newcommand{\gb}{\bar{g}}
\newcommand{\gh}{\hat{g}}
\newcommand{\sh}{\hat{s}}
\newcommand{\Sh}{\hat{S}}
\newcommand{\Sigmah}{\hat\Sigma}
\newcommand{\sigmah}{\hat\sigma}
\newcommand{\yh}{\hat{y}}

\newcommand{\X}{\ensuremath{\mathrm{X}}}
\newcommand{\R}{\ensuremath{\mathrm{R}}}
\newcommand{\p}{\ensuremath{\mathrm{P}}}

\newcommand{\osum}[1]{\sum_{#1=R}^{T-1}}
\newcommand{\osumb}[1]{\sum_{#1=R'}^{T-1}}
\newcommand{\osumc}[1]{\sum_{#1=R}^{R'-1}}
\newcommand{\omax}[1]{\max_{#1=R,\dots,T-1}}
\newcommand{\omaxb}[1]{\max_{#1=R',\dots,T-1}}
\newcommand{\omaxc}[1]{\max_{#1=R,\dots,R'-1}}
\newcommand{\oavg}[1]{\tfrac{1}{P} \osum{#1}}
\newcommand{\oavgb}[1]{\tfrac{1}{P} \osumb{#1}}
\newcommand{\oavgc}[1]{\tfrac{1}{P} \osumc{#1}}
\newcommand{\oclt}[1]{\tfrac{1}{\sqrt{P}} \osum{#1}}
\newcommand{\ocltb}[1]{\tfrac{1}{\sqrt{P}} \osumb{#1}}
\newcommand{\ocltc}[1]{\tfrac{1}{\sqrt{P}} \osumc{#1}}

\newcommand{\allcaps}[1]{\textls{\MakeUppercase{#1}}}

\newcommand{\aic}{\allcaps{AIC}}
\newcommand{\bic}{\allcaps{BIC}}
\newcommand{\brc}{\allcaps{BRC}}
\newcommand{\cdf}{\allcaps{CDF}}
\newcommand{\clt}{\allcaps{CLT}}
\newcommand{\dd}[1]{\frac{\partial}{\partial #1}}
\newcommand{\dgp}{\allcaps{DGP}}
\newcommand{\fclt}{\allcaps{FCLT}}
\newcommand{\fwe}{\allcaps{FWE}}
\newcommand{\gdp}{\allcaps{GDP}}
\newcommand{\hac}{\allcaps{HAC}}
\newcommand{\lln}{\allcaps{LLN}}
\newcommand{\lm}{\allcaps{LM}}
\newcommand{\ma}{\allcaps{MA}}
\newcommand{\mds}{\allcaps{MDS}}
\newcommand{\mse}{\allcaps{MSE}}
\newcommand{\nber}{\allcaps{NBER}}
\newcommand{\nsf}{\allcaps{NSF}}
\newcommand{\ned}{\allcaps{NED}}
\newcommand{\ols}{\allcaps{OLS}}
\newcommand{\oos}{\allcaps{OOS}}
\newcommand{\sfwe}{\allcaps{SFWE}}
\newcommand{\spa}{\allcaps{SPA}}
\newcommand{\wfwe}{\allcaps{WFWE}}

\renewcommand{\Re}{\ensuremath{\mathbb{R}}}

\renewcommand{\topfraction}{.85}
\renewcommand{\bottomfraction}{.7}
\renewcommand{\textfraction}{.15}
\renewcommand{\floatpagefraction}{.66}
\renewcommand{\dbltopfraction}{.66}
\renewcommand{\dblfloatpagefraction}{.66}
\setcounter{topnumber}{9}
\setcounter{bottomnumber}{9}
\setcounter{totalnumber}{20}
\setcounter{dbltopnumber}{9}

\frenchspacing

% Check if shell commands can be executed
\ifnum\pdfshellescape=1
% Yes, enabled
\newcommand{\VERSION}{\splice{echo `git log -1 --date=short --format=\%cd`},
  commit\splice{echo `git rev-parse --short HEAD`}}
\else
% No, disabled
\newcommand{\VERSION}{}
\fi
\date{\VERSION}

\usepackage{xr}
\externaldocument{mixedwindow}

\author{Gray Calhoun\thanks{Economics Department; Iowa State
    University; Ames, IA 50011.  Telephone: (515) 294-6271.  Email:
    \guillemotleft \protect\url{gcalhoun@iastate.edu}\guillemotright,
    web: \guillemotleft www.econ.iastate.edu/\textasciitilde
    gcalhoun\guillemotright.}\\%
  Iowa State University}

\title{Supplemental appendix for ``An asymptotically normal
  out-of-sample test based on mixed estimation windows''}

\newcommand{\WesA}[1][]{\ocltb{t}
  (F_t^{#1} - F) B^{#1} H_t^{#1}}
\newcommand{\WesB}[1][]{\ocltb{t} F (B_t^{#1} -
  B^{#1}) H_t^{#1}}
\newcommand{\WesC}[1][]{\ocltb{t}
  (F_t^{#1} - F) (B_t^{#1} - B^{#1}) H_t^{#1}}

\begin{document}
\maketitle

\noindent%
This appendix contains mathematical proofs and some supporting Lemmas
for the paper, ``An asymptotically normal out-of-sample test based on
mixed estimation windows'' \citep{Cal:15}. Define the following
additional terms:
\begin{equation*}
  F_t(\beta) = 2 (x_t'\beta - \yh_{t+1}) x_t',
\end{equation*}
$F_t = F_t(\btrue)$, $\Fh_t = F_t(\bh_t)$, $F = \E F_t$, $B = (\E x_t
x_t')^{-1}$, $B_t = (\sum_{s=1}^{t-1} x_s x_s' / (t-1))^{-1}$, and
$H_t = \sum_{s=1}^{t-1} x_s \ep_{s+1} / (t-1)$.
And let $\lVert \cdot \rVert$ denote the $L_2$ norm in $\Re^k$.

Note that Assumptions~\ref{a1} and~\ref{a3} imply that $f_t$, $g_t$,
and $F_t$ are all strong mixing of size $-r/(r-2)$ or uniform mixing
of size $-r/(2r-2)$ and are stationary with bounded $r$th moments.


\begin{rthm}{\ref{res:1}}If Assumptions~\ref{a1} -- \ref{a4} hold then
$\sqrt{P} (\fb - \E \fb(\btrue)) \to^d N(0, \Sigma)$,
with
\begin{align*}
    \Sigma &= S_{ff} + S_{fg} + S_{fg}' + 2 \, S_{gg}, &
    S_{ff}  &= \lim P \var(\fb(\btrue)), \\
    S_{fg}  &= \lim P \cov(\fb(\btrue), \gb(\btrue)), &
    S_{gg}  &= \lim P \var(\gb(\btrue)).
\end{align*}
\end{rthm}
\begin{proof}
  Let $R'$ be a new sequence such that $R' \to \infty$ as $T \to \infty$
  and $R' = o(\sqrt{P})$, and then rewrite the centered \oos\ average as
  \begin{equation}\label{eq:6}
    \sqrt{P} (\fb - \E \fb^*)
    = \ocltb{t} ((f_t - \E f_t) + (\fh_t - f_t))
      + \tfrac{1}{\sqrt{P}} \osumc{t} (\fh_t - \E f_t).
  \end{equation}
  Lemma~\ref{res:a1} ensures that the second summation is $o_p(1)$, so
  we can use a Taylor expansion to rewrite~\eqref{eq:6} as
  \begin{align*}
    \sqrt{P} (\fb - \E \fb^*)
    &= \ocltb{t} (f_t - \E f_t) + F B \ocltb{t} H_t \\
    & \quad + \WesA + \WesB \\ & \quad + \WesC + \oclt{t} w_t + o_p(1)
  \end{align*}
  where $w_t$ equals $2 (\bh_t - \btrue)' x_t x_t' (\bh_t - \btrue)$.
  Lemma~\ref{res:a4} shows that
  \begin{gather}
    \WesA \to^{p} 0 \label{eq:11} \\
    \WesB \to^{p} 0 \label{eq:12} \\
    \intertext{and}
    \WesC \to^{p} 0 \label{eq:13}
  \end{gather}
  and Lemma~\ref{res:a2} along with the \clt\ ensures that $\oclt{t}
  w_t = o_{p}(1)$. The proof that
  \begin{equation*}
    \ocltb{t} (f_t - \E f_t) + F B \ocltb{t} H_t \to N(0, \sigma^2).
  \end{equation*}
  follows the same argument as in \citet{Wes:96} and \citet{Mcc:00}.
\end{proof}

\begin{rlem}{\ref{lem:2}}If Assumptions~\ref{a1}~--~\ref{a5} hold then
\begin{equation*}
  \sigmah_1^2 \to^p \sigma^2.
\end{equation*}
If Assumptions~\ref{a1}~--~\ref{a4} hold and $\{\varepsilon_{t},
\Fs_t\}$ is an \mds\ then
\begin{equation*}
   \sigmah_2^2 \to^p \sigma^2.
\end{equation*}
\end{rlem}
\noindent%
We will only prove $\sigmah_2 \to^p \sigma$. The result for
$\sigmah_1$ is essentially the same and uses \citepos{JoD:00} Theorem
2.1 for the \hac\ equivalent of Equations~\eqref{eq:1}--\eqref{eq:7}.
\begin{proof}
  First, we can rewrite the components of the variance estimator as
  \begin{align*}
    \sh_{21} &= \oavg{t} \Big[(f_t - \E f_t) + (\fh_t - f_t) - (\fb - \E f_t)\Big]^2 \\
    \sh_{22} &= \oavg{t} \Big[(f_t - \E f_t) + (\fh_t - f_t) - (\fb - \E f_t)\Big]
                        \Big[(g_t - \E g_t) + (\gh_t - g_t) - (\gb - \E g_t)\Big]
    \intertext{and}
    \sh_{23} &= \oavg{t} \Big[(g_t - \E g_t) + (\gh_t - g_t) - (\gb + \E g_t)\Big]^2
  \end{align*}
  so $\sigmah_2 \to^p \sigma$ as long as the following hold:
  $\fb - \E \fb^* \to^p 0$,
  $\gb - \E \gb^* \to^p 0$,
  \begin{gather}
    \oavg{t} (f_t - \E f_t)^2 \to^p \lim \var(\sqrt{P} \fb^*) \label{eq:1} \\
    \oavg{t} (g_t - \E g_t)^2 \to^p \lim \var(\sqrt{P} \gb^*) \label{eq:3} \\
    \oavg{t} (f_t - \E f_t) (g_t - \E g_t) \to^p \lim \cov(\sqrt{P} \fb^*, \sqrt{P} \gb^*) \label{eq:7} \\
    \oavg{t} (\fh_t - f_t)^2 \to^p 0, \label{eq:4}
    \intertext{and}
    \oavg{t} (\gh_t - g_t)^2 \to^p 0. \label{eq:5}
  \end{gather}
  The first two results are implied by the proof of
  Theorem~\ref{res:1} and~\eqref{eq:1}, \eqref{eq:3}, and~\eqref{eq:7}
  follow from the \lln, since each summand is an $L_1$-mixingale of
  size $-1$ \citep[see, for example][Theorem 17.5]{Dav:94}, so it suffices to
  prove~\eqref{eq:4} and~\eqref{eq:5}.

  As in the proof of Theorem~\ref{res:1}, let $R'$ be a new sequence such that $R' \to \infty$ as
  $T \to \infty$ and $R' = o(\sqrt{P})$.  Straightforward algebra reveals
  that~\eqref{eq:4} holds if
  \begin{gather}
    \oavg{t} ((\bh_t - \btrue)' x_t)^4 \to^p 0 \label{eq:10}
    \intertext{and}
    \oavg{t} (x_t'(\bh_t - \btrue))^2 (2 x_t'\btrue - y_{t+1} - \yh_{t+1})^2 \to^p 0.\label{eq:14}
  \end{gather}
  The \allcaps{LHS} of~\eqref{eq:10} is bounded by
  \begin{align*}
    \oavg{t} &\|\bh_t - \btrue\|^4 \|x_t\|^4\\
    &= \oavgc{t} \|\bh_t - \btrue\|^4 \|x_t\|^4 + \oavgb{t} \|\bh_t - \btrue\|^4 \|x_t\|^4 \\
    &\leq \omaxc{t} \|\bh_t - \btrue\|^4 \,  \oavgc{t} \|x_t\|^4 + \omaxb{t} \|\bh_t - \btrue\|^4 \,  \oavgb{t} \|x_t\|^4 \\
    &= O_p(R'/P) + o_p(1)
  \end{align*}
  by Lemma~\ref{res:a2} and the \lln.
  A similar argument holds for the second term:
  \begin{align*}
    \oavg{t} (x_t'(\bh_t - \btrue))^2 &(2 x_t'\btrue - y_{t+1} - \yh_{t+1})^2 \\
    &= \oavgc{t} (x_t'(\bh_t - \btrue))^2 (2 x_t'\btrue - y_{t+1} - \yh_{t+1})^2 \\
    &\quad+ \oavgb{t} \big(x_t'(\bh_t - \btrue)\big)^2 (2 x_t'\btrue - y_{t+1} - \yh_{t+1})^2 \\
    &\leq \omaxc{t} \|\bh_t - \btrue\|^2 \oavgc{t} \|x_t (2 x_t'\btrue - y_{t+1} - \yh_{t+1})\|^2 \\
    &\quad+ \omaxb{t} \|\bh_t - \btrue\|^2 \oavgb{t} \|x_t (2 x_t'\btrue - y_{t+1} - \yh_{t+1})\|^2 \\
    &= O_p(R'/P) + o_p(1)
  \end{align*}
  again by Lemma~\ref{res:a2} and the \lln. Both terms converge to
  zero in probability by construction. The proof of~\eqref{eq:5} is similar.
\end{proof}

\section*{Supporting Results}
\stepcounter{section}
\renewcommand\thesection{\Alph{section}}

\begin{alem}\label{res:a1}
  Suppose the conditions of Theorem~\ref{res:1} hold, and define $R'$
  to be a sequence that satisfies $R' \to \infty$ as $T \to \infty$
  and $R' = o(\sqrt{P})$. Then
  \begin{equation*}
    \ocltc{t} (\fh_t - \E f_t) \to^p 0.
  \end{equation*}
\end{alem}

\begin{proof}
  We can rewrite this summation as
  \begin{multline*}
    \ocltc{t} (\fh_t - \E f_t) = \ocltc{t} (f_t - \E f_t) + \\
    \ocltc{t} (4 x_t'\btrue - 2 y_{t+1} - 2 \yh_{i,t+1}) x_t'(\bh_t - \btrue)
    + \ocltc{t} (x_t'\bh_t - x_t'\btrue)^2.
  \end{multline*}
  Each of these individual summations can be shown to converge to
  zero in probability. First,
  \begin{equation*}
    \E \Big\lvert \ocltc{t} (f_t - \E f_t) \Big\rvert
    \leq \ocltc{t} \E\lvert f_t - \E f_t \rvert
    = O(R'/\sqrt{P}).
  \end{equation*}
  Also,
  \begin{align*}
    \Big\lvert \ocltc{t} & (4 x_t'\btrue - 2 y_{t+1} - 2 \yh_{i,t+1}) x_t'(\bh_t - \btrue) \Big\rvert \\
    &\leq \ocltc{t} \big\lVert (4 x_t'\btrue - 2 y_{t+1} - 2 \yh_{i,t+1}) x_t \big\rVert
    \omaxc{t} \lVert  \bh_t - \btrue \rVert \\
    & = O_p(R'/\sqrt{P})
  \end{align*}
  and
  \begin{equation*}
    \Big\lvert \ocltc{t} (x_t'\bh_t - x_t'\btrue)^2 \Big\rvert
    \leq \ocltc{t} \lVert x_t \rVert^2 \omaxc{t} \lVert \bh_t - \btrue \rVert^2
    = O_p(R'/\sqrt{P}).
  \end{equation*}
  by Lemma~\ref{res:a2} and the \lln. Since $R'/\sqrt{P} \to 0$ by
  construction, this completes the proof.
\end{proof}

\begin{alem}\label{res:a2}
  Suppose $a \in [0,1/2)$ and Assumptions~\ref{a1}~--~\ref{a4}
  hold, and let $R'$ be a sequence such that $R' \to \infty$ as $T \to
  \infty$ and $R' = o(\sqrt{P})$. Then
  \begin{enumerate}
  \item $\omaxb{t} | (t-1)^a H_t | \to^p 0$,
  \item $\omaxc{t} | (t-1)^a H_t | = O_p(1)$,
  \item $\omaxb{t} | B_t - B | \to^p 0$,
  \item $\omaxc{t} | B_t - B | = O_p(1)$,
  \item $\omaxb{t} | (t-1)^a(\bh_t - \btrue) | \to^{p} 0$, and
  \item $\omaxc{t} | (t-1)^a(\bh_t - \btrue) | = O_p(1)$,
  \end{enumerate}
  where the absolute value is taken as the largest of the
  element-by-element absolute values.
\end{alem}

\noindent%
To streamline the
presentation, we'll assume in these proofs that $x_t$ is a scalar.
\begin{proof}
  We will prove each part in order.
  \begin{enumerate}
  \item Our assumptions ensure that $x_t \ep_{t+1}$ is $L_2$-mixingale
    of size $-1/2$ \citep[see Theorem 17.5 of][]{Dav:94}; let $c_t$
    and $\zeta_k$ denote its mixingale
    constants and coefficients. Note that, for any $b$, $t^{b} x_t \ep_{t+1}$ is
    also an $L_2$-mixingale array with constants $t^{b} c_s$ and
    coefficients $\zeta_k$, since
    \begin{align*}
      \| \E_{t-k} t^{b} x_t \ep_{t+1} \| &= t^{b} \| \E_{t-k} x_t \ep_{t+1} \| \\
      &\leq (t^{b} c_t) \zeta_k
    \end{align*}
    and
    \begin{align*}
      \| t^{b} x_t \ep_{t+1} - t^{b} \E_{t+k} x_t \ep_{t+1} \| &= t^{b} \|  x_t \ep_{t+1} - \E_{t+k} x_t \ep_{t+1} \| \\
      &\leq (t^{b} c_t) \zeta_{k+1}.
    \end{align*}

    Let $\delta$ be a positive number less than $1/2 - \alpha$, so
    \begin{align*}
      \E\Bigg[\omaxb{t} \Big|(t-1)^{a-1} & \sum_{s=1}^{t-1} x_s \ep_{s+1} \Big|^2\Bigg] \\
      &\leq (R'-1)^{-2\delta} \E\Bigg[\omaxb{t} \Big| \sum_{s=1}^{t-1} x_s \ep_{s+1} (s-1)^{a-1+\delta} \Big|^2\Bigg] \\
      &\leq (R'-1)^{-2\delta} O(1) \sum_{s=1}^{T-1} (s-1)^{2(a - 1 + \delta)}
    \end{align*}
    where the second inequality follows from \citepos{Mcl:75} maximal
    inequality (also available as \citealp{Dav:94}, Theorem 16.9 and
    Corollary 16.10). The summation converges to a constant as $T \to
    \infty$ and $(R'-1)^{-2\delta} \to 0$, completing the proof.

  \item Now $t^{a-1} x_t \ep_{t+1}$ is an $L_2$-mixingale of size
    $-1/2$ and we can again use \citepos{Mcl:75} maximal inequality to get
    \begin{align*}
      \E \Big\lvert \omaxc{t} \Big((t-1)^{a - 1} \sum_{s=1}^{t-1} x_s \ep_{s+1} \Big)^2 \Big\rvert
      &\leq \E \Big\lvert \omax{t} \Big(\sum_{s=1}^{t-1} s^{a-1} x_s \ep_{s+1} \Big)^2 \Big\rvert \\
      &= O(1) \sum_{s=1}^{R'-1} s^{2a - 2}
    \end{align*}
    which converges to a finite limit.
  \item The same argument used in Part 1 implies that $\omaxb{t} |
    B_t^{-1} - B^{-1}| \to^p 0$. Since matrix inversion is continuous,
    the result follows.
  \item Holds by Assumptions~\ref{a1} and~\ref{a3}.
  \item We have
    \begin{align*}
      \omaxb{t} | (t-1)^a (\hat{\beta}_t - \btrue) |
      &\leq \omaxb{t} |\hat{B}_t - B|
      \omaxb{t} \Big|(t-1)^{a-1} \sum_{s=1}^{t-1} x_s \ep_{s+1} \Big| \\
      &\quad + \omaxb{t} \Big| B (t-1)^{a-1} \sum_{s=1}^{t-1} x_s \ep_{s+1} \Big|
    \end{align*}
    and both terms converge to zero in probability by Parts 1 and 3.
  \item Similar to the previous argument, we have
    \begin{align*}
      \omaxc{t} | (t-1)^a (\hat{\beta}_t - \btrue) |
      & \leq \omaxc{t} | \hat{B}_t - B | \omaxc{t} \Big|(t-1)^{a-1} \sum_{s=1}^{t-1} x_s \ep_{s+1} \Big| \\
      &\quad + \omaxc{t} \Big| B (t-1)^{a-1} \sum_{s=1}^{t-1} x_s \ep_{s+1} \Big|.
    \end{align*}
    Both terms are $O_p(1)$ by Parts 2 and 4. \qedhere
  \end{enumerate}
\end{proof}

\begin{alem}\label{res:a4}
  Under the conditions of Theorem~\ref{res:1}, Equations
  \eqref{eq:11}--\eqref{eq:13} hold.
\end{alem}

\begin{proof}
We can write
\begin{equation*}
  \Big\lvert \WesA \Big\rvert \leq
  \Big\lVert \ocltb{t} (F_t - F) B \Big\rVert \;
  \omaxb{t} \lVert H_t \rVert.
\end{equation*}
From Lemma~\ref{res:a2}, $\omaxb{t} \lVert H_t \rVert \to^p
0$. \citepos{Jon:97} \clt\ implies that $\ocltb{t} (F_t - F) =
O_p(1)$, establishing~\eqref{eq:11}. The proofs of \eqref{eq:12}
and~\eqref{eq:13} are similar.
\end{proof}

\bibliography{texextra/references}
\end{document}
