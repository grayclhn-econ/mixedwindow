\documentclass[final]{beamer}
  \mode<presentation>
  {
% you can chose your theme here:
% \usetheme{Berlin}
% further beamerposter themes are available at 
% http://www-i6.informatik.rwth-aachen.de/~dreuw/latexbeamerposter.php
}
\usepackage{type1cm}
\usepackage{calc} 
\usepackage{xcolor}
\usepackage[orientation=landscape,size=a0,scale=1.25]{beamerposter}
\setlength{\paperwidth}{40in}
\setlength{\paperheight}{30in}
\setlength{\textwidth}{0.98\paperwidth}
\setlength{\textheight}{0.98\paperheight}

\newcommand{\totalsims}{2000} \newcommand{\testsize}{10}
\newcommand{\empiricalcriticalvalue}{2.55}
\newcommand{\naivecriticalvalue}{1.28}
\newcommand{\size}{10}
\newcommand{\windowlength}{10}
\newcommand{\nmod}{30}
\newcommand{\empiricaldraws}{1999}
\newcommand{\empiricaltable}{% latex table generated in R 3.2.0 by xtable 1.7-4 package
% Mon Feb  1 13:15:49 2016
\begin{tabularx}{\textwidth}{lCCC}
  \toprule   & value & naive & corrected \\ 
  \midrule book to market CT & $\enskip2.04$ & sig. &  \\ 
  long term rate CT & $\enskip1.64$ & sig. &  \\ 
  median & $\enskip1.59$ & sig. &  \\ 
  long term rate & $\enskip1.56$ & sig. &  \\ 
  book to market & $\enskip1.41$ & sig. &  \\ 
  dividend yield CT & $\enskip1.30$ & sig. &  \\ 
  dividend yield & $\enskip1.26$ &  &  \\ 
  stock variance CT & $\enskip1.22$ &  &  \\ 
  dividend payout ratio CT & $\enskip1.18$ &  &  \\ 
  average & $\enskip1.04$ &  &  \\ 
  dividend price ratio & $\enskip0.95$ &  &  \\ 
  treasury bill CT & $\enskip0.89$ &  &  \\ 
  dividend price ratio CT & $\enskip0.82$ &  &  \\ 
  default yield spread CT & $\enskip0.70$ &  &  \\ 
  net equity & $\enskip0.70$ &  &  \\ 
  net equity CT & $\enskip0.69$ &  &  \\ 
  earnings price ratio CT & $\enskip0.65$ &  &  \\ 
  dividend payout ratio & $\enskip0.64$ &  &  \\ 
  treasury bill & $\enskip0.53$ &  &  \\ 
  stock variance & $\enskip0.50$ &  &  \\ 
  inflation CT & $\enskip0.20$ &  &  \\ 
  default return spread & $\enskip0.16$ &  &  \\ 
  default return spread CT & $\enskip0.12$ &  &  \\ 
  default yield spread & $\enskip0.09$ &  &  \\ 
  inflation & $\!\!-0.09$ &  &  \\ 
  term spread CT & $\!\!-0.29$ &  &  \\ 
  term spread & $\!\!-0.43$ &  &  \\ 
  earnings price ratio & $\!\!-0.56$ &  &  \\ 
  long term yield & $\!\!-0.73$ &  &  \\ 
  long term yield CT & $\!\!-0.89$ &  &  \\ 
   \bottomrule \end{tabularx}
}

\frenchspacing

\usepackage[sort,round,comma]{natbib}
\newcommand\citepos[2][]{\citeauthor{#2}'s \citeyearpar[#1]{#2}}
\newcommand\poscw{\citeauthor{ClW:06}'s \citeyearpar{ClW:06,ClW:07}}
\bibliographystyle{abbrvnat}

\usepackage{amsmath,amsthm, amssymb, latexsym,tabularx,booktabs}
\newcolumntype{C}{>{\centering\arraybackslash}X}
\usepackage[english]{babel}
\usepackage[latin1]{inputenc}

\newtheorem{thm}{Theorem}
\newtheorem{lem}[thm]{Lemma}
\newtheorem{claim}[thm]{Claim}
\newtheorem{cor}[thm]{Corollary}
\newtheorem{lema}{Lemma}[section]
\newtheorem{alg}{Algorithm}
\newtheorem{asmp}{Assumption}[section]

\theoremstyle{definition}

\newtheorem{defn}{Definition}
\newtheorem{rem}{Remark}

\DeclareMathOperator*{\argmin}{arg\,min}
\DeclareMathOperator{\E}{E}
\DeclareMathOperator{\var}{var}
\DeclareMathOperator{\cov}{cov}
%\DeclareMathOperator{\vec}{vec}
\DeclareMathOperator{\vech}{vech}

\DeclareMathOperator{\pr}{Pr}

\newcommand{\btrue}[1][]{\if#1*\hat\beta_{T+1}\else\beta_0\fi}

\newcommand{\X}{\ensuremath{\mathrm{X}}}
\newcommand{\R}{\ensuremath{\mathrm{R}}}
\newcommand{\p}{\ensuremath{\mathrm{P}}}

\newcommand{\osum}[1]{\sum_{#1=R+1}^T}
\newcommand{\oavg}[1]{\tfrac{1}{P} \osum{#1}}
\newcommand{\oclt}[1]{\tfrac{1}{\sqrt{P}} \osum{#1}}

\newcommand{\aic}{\textsc{aic}}
\newcommand{\bic}{\textsc{bic}}
\newcommand{\brc}{\textsc{brc}}
\newcommand{\cdf}{\textsc{cdf}}
\newcommand{\clt}{\textsc{clt}}
\newcommand{\dd}[1]{\frac{\partial}{\partial #1}}
\newcommand{\dgp}{\textsc{dgp}}
\newcommand{\fclt}{\textsc{fclt}}
\newcommand{\fwe}{\textsc{fwe}}
\newcommand{\gdp}{\textsc{gdp}}
\newcommand{\hac}{\textsc{hac}}
\newcommand{\lln}{\textsc{lln}}
\newcommand{\ma}{\textsc{ma}}
\newcommand{\mds}{\textsc{mds}}
\newcommand{\ned}{\textsc{ned}}
\newcommand{\ols}{\textsc{ols}}
\newcommand{\oos}{\textsc{oos}}
\newcommand{\sfwe}{\textsc{sfwe}}
\newcommand{\spa}{\textsc{spa}}
\newcommand{\wfwe}{\textsc{wfwe}}

\renewcommand{\Re}{\ensuremath{\mathbb{R}}}


\setbeamertemplate{frametitle}[default][center]
\setbeamertemplate{items}[circle]
\setbeamertemplate{navigation symbols}{}
\usecolortheme{dove}

% \setbeamercolor{frametitle}{fg=black}
% \setbeamercolor{sec title}{parent=titlelike}
% \setbeamertemplate{navigation symbols}{}

\setbeamerfont{sec title}{parent=title}
\newcommand*{\secpage}{\usebeamertemplate*{sec page}}
\setbeamercolor{frametitle}{fg=black}

\begin{document}
\begin{frame}{\\\centering \textbf{\huge An asymptotically normal out-of-sample
    test of equal predictive accuracy for nested models}}
  \centering

  Gray Calhoun, Iowa State University Economics Department

  Sept. 26, 2013, NBER-NSF Time Series Conference

  \begin{columns}[t]
    \begin{column}{.23\textwidth}

      \begin{block}{\textbf{Abstract / overview}}
        \begin{itemize}
        \item[] This paper proposes a modification of
          \citepos[\textit{J.  Econom.}]{ClW:07} adjusted
          out-of-sample $t$-test.  The alternative model is still
          estimated with a fixed-length rolling window, but the
          benchmark is estimated with a recursive window. The
          resulting statistic is asymptotically normal even when the
          models are nested.  Moreover, the alternative model can be
          estimated using common model selection methods, such as the
          AIC or BIC.  This paper also presents a method to compare
          multiple models simultaneously while controlling familywise
          error, and substantially improves existing block bootstrap
          methods for out-of-sample statistics.  This procedure is
          then used to analyze
          \citepos[\textit{Rev. Finan. Stud.}]{GoW:08} excess returns
          dataset.
        \end{itemize}
      \end{block}
      \bigskip
      \begin{block}{\textbf{Test statistic for out-of-sample model
          comparisons (Theorem~1)}}
        \begin{itemize}
        \item Standard out-of-sample (OOS) model comparison:
          \begin{itemize}
          \item $R+1$ observations used as training/estimation sample
          \item $P$ used as test sample
          \item $T+1$ total observations ($T = R + P$)
          \end{itemize}
        \item $\hat{y}_{0t}$ and $\hat{y}_{1t}$ forecast the variable
          $y_t$
        \item $\hat{y}_{0,t+1}$, is estimated using OLS with a
          recursive window: $\hat{y}_{0,t+1} = x_t'\hat{\beta}_t$
          (benchmark model)
        \item $\hat{y}_{1t}$, is estimated using a rolling window of
          fixed length $R$ (less than $T$):
          \begin{equation*}
            \hat{y}_{1,t+1} = \psi(y_t,z_t,\dots,y_{t-R+1}, z_{t-R+1})
          \end{equation*}
          where $\psi$ is a known function
        \item Standard moment, dependence, and positive definiteness
          conditions hold (see paper)
        \item Null hypothesis: $y_t - \hat{y}_{0t}(\btrue)$
          is a martingale difference sequence with respect to the
          filtration $\mathcal{F}_t = \sigma((y_t, z_{t}), (y_{t-1},
          z_{t-1}),\dots)$
        \item \textbf{Theorem 1:} $\sqrt{P} \bar f / \hat\sigma \to^d
          N(0,1)$, where
          \begin{multline*}
            \bar f = \tfrac{1}{P} \sum_{t=R+1}^T \Big[(y_{t+1} - \hat y_{0,t+1})^2
            - (y_{t+1} - \hat{y}_{1,t+1})^2
            \\ + (\hat y_{0,t+1} - \hat{y}_{1,t+1})^2 \Big],
          \end{multline*}
          formula for the variance defined in the paper
        \item Presented for one step ahead forecasting but can be
          extended to longer horizons
        \item Can be used to test for zero correlation (instead of
          MDS) or forecast optimality with different loss (other than
          MSE) with minor modifications
        \end{itemize}
      \end{block}
    \end{column}

    \begin{column}{.23\textwidth}
      \begin{block}{\textbf{Comparing several models (Theorem~2)}}
        \begin{itemize}
        \item Extends results in Theorem~1 to comparisons of many
          models
        \item $m$ alternative forecasting models,
          $\hat{y}_{1t},\dots, \hat{y}_{mt}$
          \item Data matrices $\X_R = (X_1,\dots,X_R)'$
            and $\X_{P} = (X_{R+1},\dots,X_{T-1})$ with
            \begin{equation*}
              X_t = \begin{cases}
                (y_{t+1}, x_t')' & t \leq R \\
                (y_{t+1}, x_t', \hat{y}_{1,t+1}, \dots, \hat{y}_{m,t+1})' & t > R.
              \end{cases}
            \end{equation*}
        \item Bootstrap procedure:
          \begin{enumerate}
          \item Draw $B$ samples of $P$ observations from $\X_{P}$
            with block bootstrap (stationary, circular, etc.)
          \item Denote each sample as $\X_{P, l}^{*}$ and let
            $\X_l^{*} = [\X_{R}', \X_{P,l}^{*\prime}]'$
          \item Estimate $\bar{f}^{*}_{li}$ and
            $\hat{\sigma}_{li}^{*}$ as in Theorem~1 for each bootstrap
            sample, $\X^{*}_l$, and each alternative model,
            $\hat{y}_{\ell t}$
          \end{enumerate}
        \item Test procedure (Theorem 2):
          \begin{enumerate}
          \item Define $M_0 = \emptyset$ and, for $j = 1,\dots,m$, let
            $M_j = \{\ell : \tfrac1{\hat\sigma_l} \bar{f}_{\ell} >
            \hat{d}_j\}$, where $\hat{d}_j$ is the $1-\alpha$ quantile
            of $\max_{\ell \notin M_{j-1}}
            \tfrac{1}{\hat{\sigma}_\ell^{*}}(\bar{f}_{\ell}^{*} -
            \bar{f_\ell}(\btrue[*]))$.
          \item Reject all of the models $\ell$ with $\ell \in
            \bigcup_{j=1}^m M_j$
          \end{enumerate}
        \item $\bar f_\ell$ is Theorem 1's statistic for the $\ell$th
          alternative model
        \item Controls familywise error rate at level $\alpha$
        \item Similar assumptions on each model as in Theorem 1
        \end{itemize}
      \end{block}

      \bigskip

      \begin{block}{\textbf{Bootstrap for out-of-sample statistics (Theorem 3)}}
        \begin{itemize}
        \item This section concerns \emph{general} OOS comparisons
        \item Standard moment, dependence, positive definiteness and
          smoothness conditions (still) assumed
        \item \textbf{Theorem 3:}
          Then, for all $\epsilon > 0$,
          \begin{multline*}
            \pr[\sup_x | \pr^*[\sqrt{P} (\bar{f}^* - \bar{f}(\tilde{\beta}_T))
            \leq x] \\ - \pr[\sqrt{P}( \bar{f} - \E \bar{f}(\btrue)) \leq x] | >
            \epsilon] \to 0
          \end{multline*}
          \begin{itemize}
          \item $\bar f^*$ is the bootstrapped OOS statistic
            (stationary, circular, or moving blocks)
          \item $\bar f$ is the OOS statistic
          \item $\E \bar f(\beta_0)$ is the population quantity of
            interest
          \item $\bar f(\tilde \beta_T)$ is the OOS average using
            the \emph{full-sample} parameter estimate
          \end{itemize}
        \item $R, P \to \infty$ as $T \to \infty$ and the (expected)
          block length $b$ satisfies $b \to \infty$ and $\frac{b}{P}
          \to 0$
        \item Other nonparametric bootstraps for OOS statistics
          require $(P/R) \log \log R \to 0$ or redefining
          $\hat\beta_t^*$
        \item Parametric bootstrap requires that the benchmark model
          be correctly specified
        \end{itemize}
      \end{block}
    \end{column}

    \begin{column}{.23\linewidth}
      \begin{block}{\textbf{Monte Carlo simulations}}
        \smallskip
        \bigskip
        {
          \footnotesize
          % latex table generated in R 3.2.0 by xtable 1.8-0 package
% Tue Feb  2 03:47:02 2016
\begin{tabularx}{\textwidth}{lCCCCC}
  \toprule Sim. type & R & P & Pr[CW~roll.] & Pr[CW~rec.] & Pr[new] \\ 
  \midrule size & $120$ & $120$ & $\enskip7.3$ & $\enskip7.8$ & $\enskip\enskip7.6$ \\ 
   &  & $240$ & $\enskip5.5$ & $\enskip5.5$ & $\enskip\enskip6.2$ \\ 
   &  & $360$ & $\enskip7.5$ & $\enskip6.2$ & $\enskip\enskip7.7$ \\ 
   &  & $720$ & $\enskip8.5$ & $\enskip5.4$ & $\enskip\enskip7.2$ \\ 
   & $240$ & $120$ & $\enskip7.2$ & $\enskip7.2$ & $\enskip\enskip7.7$ \\ 
   &  & $240$ & $\enskip6.3$ & $\enskip6.5$ & $\enskip\enskip7.1$ \\ 
   &  & $360$ & $\enskip6.8$ & $\enskip5.9$ & $\enskip\enskip6.8$ \\ 
   &  & $720$ & $\enskip7.0$ & $\enskip5.9$ & $\enskip\enskip7.3$ \\ 
  power (stable) & $120$ & $120$ & $26.2$ & $30.0$ & $\enskip29.2$ \\ 
   &  & $240$ & $39.2$ & $47.2$ & $\enskip42.4$ \\ 
   &  & $360$ & $47.3$ & $59.8$ & $\enskip51.1$ \\ 
   &  & $720$ & $66.8$ & $82.3$ & $\enskip73.1$ \\ 
   & $240$ & $120$ & $34.5$ & $36.1$ & $\enskip34.1$ \\ 
   &  & $240$ & $45.9$ & $50.1$ & $\enskip46.9$ \\ 
   &  & $360$ & $56.7$ & $63.8$ & $\enskip56.9$ \\ 
   &  & $720$ & $78.2$ & $87.0$ & $\enskip78.7$ \\ 
  power (breaks) & $120$ & $120$ & $25.9$ & $29.9$ & $\enskip62.2$ \\ 
   &  & $240$ & $30.1$ & $31.0$ & $\enskip87.4$ \\ 
   &  & $360$ & $35.5$ & $32.9$ & $\enskip96.5$ \\ 
   &  & $720$ & $46.1$ & $38.2$ & $\enskip99.8$ \\ 
   & $240$ & $120$ & $28.1$ & $30.6$ & $\enskip58.2$ \\ 
   &  & $240$ & $37.6$ & $36.1$ & $\enskip87.7$ \\ 
   &  & $360$ & $43.1$ & $39.0$ & $\enskip97.2$ \\ 
   &  & $720$ & $56.9$ & $42.5$ & $100.0$ \\ 
   \bottomrule \end{tabularx}

        }
        \begin{itemize}
        \item Size and power of the OOS tests at \testsize\%
          confidence, based on \totalsims\ simulations
        \item Shows fraction of simulations for which each test rejects:
          \begin{itemize}
          \item Pr[\textsc{cw} roll.]: Clark and West's (2007)
            rolling-window
          \item Pr[\textsc{cw} rec.]: Clark and West's (2007)
            recursive-window
          \item Pr[new]: Our new mixed-strategy test statistic
          \end{itemize}
        \item DGP given by
          \begin{align*}
            y_t &= \gamma_{1t} + \gamma_{2t} z_{t-1} + e_t \\\\
            \gamma_t &=
            \begin{cases}
              (0.5, 0)    & \text{size simulations} \\
              (0.5, 0.35) & \text{power (stable)} \\
              (-0.5, 0)    & t \leq \tfrac{T}{2} \quad \text{power (break)} \\
              (1, 0.35) & t > \tfrac{T}{2} \quad \text{power (break)}
            \end{cases}\\\\
            z_t &= 0.15 + 0.95 z_{t-1} + v_t \\\\
            (e_t, v_t)' &\sim iid\ N\Bigg(\begin{pmatrix} 0 \\ 0
            \end{pmatrix}
            , \begin{pmatrix} 18 & -
              0.5 \\ -0.5 & 0.025 \end{pmatrix}\Bigg)
          \end{align*}
        \item Benchmark regresses $y_t$ on a constant
        \item Alternative regresses $y_t$ on a constant and $z_{t-1}$ (OLS)
        \item Mimics the asset pricing application studied later in the paper
          \citep{GoW:08}
        \end{itemize}
      \end{block}
    \end{column}

    \begin{column}{.23\linewidth}
      \begin{block}{\textbf{Application to equity premium predictability}}
        \smallskip
        \bigskip
        {
          \footnotesize
          \empiricaltable
        }

        \begin{itemize}
        \item OOS comparison of equity premium prediction models
          based on \citet{GoW:08}
        \item Benchmark is the recursive sample mean of the equity
          premium and each alternative model is a constant and single
          lag of the variable listed in the \emph{predictor} column
          (\emph{CT} indicates \citepos{CaT:08} correction to forecasts)
        \item Dataset begins in 1927 and ends in 2009, annual data
        \item \emph{Value} column lists the value of this paper's
          OOS statistic
        \item \emph{Naive} column indicates whether the statistic is
          significant at standard critical values
        \item \emph{Corrected} column indicates significance using
          critical values that account for the number of models
        \end{itemize}
      \end{block}

      \begin{block}{\textbf{Contact Information}}
        \begin{itemize}
        \item Economics Department; Iowa State University
        \item Telephone: (515) 294-6271.
        \item Email: gcalhoun@iastate.edu
        \item Web: http://gray.clhn.co
        \end{itemize}
      \end{block}

      
    \end{column}
  \end{columns}
\end{frame}
\begin{frame}<beamer:0>
    \bibliography{texextra/AllRefs}
\end{frame}
\end{document}


%%% Local Variables: 
%%% mode: latex
%%% TeX-PDF-mode: t

%  LocalWords:  NBER Econom ClW AIC familywise Finan GoW dataset MDS
%  LocalWords:  optimality MSE nonparametric DGP iid beamer texextra AllRefs
