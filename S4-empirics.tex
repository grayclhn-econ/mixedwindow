\section{Empirical Illustration}\label{sec:3}

This section demonstrates the use of our new statistic by revisiting
\citepos{GoW:08} study of excess stock returns.  Goyal and Welch argue
that many variables thought to predict excess returns (measured as the
difference between the yearly log return of the S\&P 500 index and the
T-bill interest rate) on the basis of in-sample evidence fail to do so
out-of-sample.  To show this, Goyal and Welch look at the forecasting
performance of models using a lag of the variable of interest, and
show that these models do not significantly outperform the excess
return's recursive sample mean.

Here, we conduct the same analysis, but using this paper's \mds\ test.
The benchmark model is the excess return's sample mean (as in the
original) and the alternative models are of the form
\begin{equation*}
  \mathit{excess~return}_{t+1} = \beta_{0} + \beta_{1}\
  \mathit{predictor}_{t} + \ep_{t+1},
\end{equation*}
where $\beta_{0}$ and
$\beta_{1}$ are estimated by \ols\ using a \windowlength-year window.
The predictors used are listed in the \emph{predictor} column of
Table~\ref{tab:em1}. \citep[See][for a detailed description of the
variables.]{GoW:08}  We also consider \citepos{CaT:08} proposed
correction to the models, that the forecasts be bounded below by zero
since negative forecasts are incredible, as well as two simple
combination forecasts, the mean and the median (over both the original
and the non-negative forecasts).  The data set is annual data
beginning in 1927 and ending in 2009, and the rolling window uses
\windowlength\ observations.%
\footnote{This statistical analysis was conducted in R \citep{R} and
  uses the MASS \citep[7.3-22]{VeR:02} package.} %

\begin{table}[tb!]
  \centering
  \empiricaltable
\caption{Results from \oos\ comparison of equity premium prediction
  models; the benchmark is the recursive sample mean of the equity
  premium and each alternative model is a constant and single lag of
  the variable listed in the \emph{predictor} column.  The dataset begins
  in 1927 and ends in 2009 and is annual data. The \emph{value} column
  lists the value of this paper's \oos\ statistic, the \emph{naive}
  column indicates whether the statistic is significant at standard
  critical values, and the \emph{corrected} column indicates significance
  using critical values that
  account for the number of models.  See Section~\ref{sec:3} for details.}
\label{tab:em1}
\end{table}

Table~\ref{tab:em1} presents the results for each model.  The column
\emph{value} gives the value of the test statistic for each model,
while \emph{naive} indicates whether the statistic is greater than the
standard 10\% critical value (\naivecriticalvalue). Three predictors
are significant at the naive critical values for both the original and
bounded forecasts: the dividend yield, long term interest rate, and
book to market ratio, and the median forecast is significant as well.
This could suggest that excess returns are not an \mds\ and that
information in these three variables is useful for predicting returns.

However, we know that this is an extremely optimistic assessment of
the models' performance. We are conducting \nmod\ simultaneous
hypothesis tests, so it is likely that some will reject by chance.
There are several approaches that could accommodate this multiplicity and a full
treatment is beyond the scope of this paper, however, it is
straightforward to use our results to
derive a valid critical value similar to~\citet{Whi:00}.

Let $\fb_i$ be the \oos\ statistic associated with the $i$th alternative
forecast, $\yh_{i,t+1}$. The arguments underlying our results apply essentially unchanged to
multivariate $f_t$, so the continuous mapping theorem implies that
\begin{equation*}
  \max_{i=1,\dots,\nmod} \sqrt{P} \fb_i/\sigmah_{2i} \to^d \max_{i=1,\dots,\nmod} W_i,
\end{equation*}
where $W \sim N(0, V)$ and $V$ is the $\nmod
\times \nmod$ correlation matrix with elements
\begin{equation*}
  V_{ij} = \lim \frac{\cov(\fb_i, \fb_j)}{\var(\fb_i)^{1/2} \var(\fb_j)^{1/2}}.
\end{equation*}

To estimate $V$, we use the correlation matrix associated with the
multivariate analogue of $\sigmah_2$,
\[
\oavg{t}\Big[(\fh_t - \fb)(\fh_t - \fb)' + (\fh_t - \fb)(\gh_t - \gb)'
+ (\gh_t - \gb)(\fh_t - \fb)' + 2 (\gh_t - \gb)(\gh_t - \gb)' \Big]
\]
where $\fh_t$ and $\gh_t$ are vectors with $i$th elements
\begin{gather*}
  \fh_{it} = (y_{t+1} - x_t'\bh_t)^2 - (y_{t+1} - \yh_{i,t+1})^2 + (x_t'\bh_t - \yh_{i,t+1})^2
  \intertext{and}
  \gh_{it} = 2\, \Bigg[\oavg{s} (x_s'\bh_s - \yh_{i,s+1}) x_s'\Bigg]\,
            \Bigg[\tfrac{1}{T-1} \sum_{s=1}^{T-1} x_s x_s'\Bigg]^{-1} x_t \eph_{t+1}
\end{gather*}
respectively. Call this estimate $\hat V$ and let $\hat c$ denote the
$0.90$ quantile of the distribution of $\max_i \hat W_i$, with
$\hat W \sim N(0, \hat V)$.  Then
\begin{equation*}
  \limsup_{T \to \infty} \Pr\Big[\max_{i=1,\dots,\nmod} \sqrt{P} \fb_i / \sigmah_{2i} > \hat c\Big] \leq 0.10,
\end{equation*}
under the null hypothesis that excess returns are an \mds\ with
respect to all of the information contained in the variables listed in
Table~\ref{tab:em1}, making $\hat c$ an asymptotically valid critical
value.%
\footnote{\citet{Han:05} makes the point that multiple one-sided
  comparisons can have poor power if irrelevant predictors are
  included in these tests and proposes a threshold for discarding very
  poor forecasts. His threshold is well below our worst performing
  model, so this issue is not a concern here.} %

We calculate $\hat c$ by generating \empiricaldraws\ draws from
$N(0,\hat V)$, giving a value of \empiricalcriticalvalue, and the
\emph{corrected} column of Table~\ref{tab:em1} denotes the models that
remain significant at 10\% with this critical value. Using this
critical value, none of the predictors are significant, which
gives additional support to \citepos{GoW:08} conclusion that excess
returns are unpredictable and also demonstrates the importance of
correcting for multiplicity in these studies.

%%% Local Variables:
%%% mode: latex
%%% TeX-master: "mixedwindow"
%%% TeX-command-extra-options: "-shell-escape"
%%% End:
