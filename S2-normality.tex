\section[Asymptotic normality]{Theoretical results supporting the asymptotically normal \oos\ statistic}
\label{sec:1}

This section presents the new \oos\ statistic; first we give an
informal motivation of the statistic, then present the paper's key
assumptions in Section~\ref{sec:1a} and present our formal theoretical
results in Section~\ref{sec:1b}.

Suppose for now that a researcher is interested in
predicting the target variable $y_{t+1}$ with a vector of regressors
$x_t$, that $v_t$ is another random process that is believed to
potentially contain information about $y_{t+1}$, and that
$(y_t, x_t, v_t)$ is stationary and weakly dependent.
In addition,
let $\btrue = (\E x_t x_t')^{-1} \E x_t y_{t+1}$ be the pseudotrue
coefficient for the regression of $y_{t+1}$ on $x_t$ and define
$\ep_{t+1} = y_{t+1} - x_t'\btrue$.  If this linear model is
correctly specified, then $\ep_{t+1}$ is an \mds\ with respect
to $\sigma((x_t, v_t, y_t), (x_{t-1}, v_{t-1}, y_{t-1}),\dots)$
and we can see immediately that
\begin{equation}
  \label{eq:4}
  \oclt{t} \ep_{t+1} (v_t - x_t'\btrue)
\end{equation}
obeys an \mds\ \clt\ and is asymptotically normal as $P \to \infty$,%
\footnote{This claim assumes that the asymptotic variance of the
  sample average is uniformly positive, a requirement that we will
  address in Section~\ref{sec:1b}.} %
with $R$ an arbitrary starting value
and $P = T - R$.

Straightforward algebra \citep{ClW:07} shows that
\begin{equation}
  \label{eq:5}
  \tfrac{1}{\sqrt{P}} \osum{t} \ep_{t+1} (v_t -
  x_t'\btrue) = \tfrac{1}{2 \sqrt{P}} \osum{t} \Big[(y_{t+1} -
  x_t\btrue)^2 - (y_{t+1} - v_t)^2 + (x_t'\btrue - v_t)^2 \Big]
\end{equation}
almost surely.
\citet{ClW:06,ClW:07} base their \oos\ statistics on the \allcaps{RHS} of
Equation~\eqref{eq:5}, but use a second forecast of $y_{t+1}$ as
$v_t$. (Call it $\yh_{t+1}$.) They use a rolling window of length
$R$ to estimate $\yh_{t+1}$,%
\footnote{Making $\yh_{t+1}$ a function of $y_t, x_{t-1}, z_{t-1}
\dots, y_{t-R+1}, x_{t-R}$ and $z_{t-R}$, where $z_t$ is another weakly
dependent random process} %
and $R$ is kept finite as $T \to \infty$ so that
$\yh_{t+1}$ inherits the weak dependence properties of the
variables used to estimate it. Using a finite window prevents
the degeneracy that can arise when comparing nested models out-of-sample (see
\citealp{ClM:01}, and \citealp{Mcc:07}), so the conditional variance
of the \oos\ average remains positive and the average obeys a \clt.%
\footnote{This approach was first introduced by \citet{GiW:06}.} %

\citet{ClW:06,ClW:07} propose using this as a test of whether the benchmark is correctly specified.
In their 2006 paper, Clark and West assume that the
coefficients on the benchmark model, $\btrue$, are zero under the null, making
$\ep_{t+1}$ observed directly. This restriction is relaxed in
their 2007 paper, where $\btrue$ is unknown and estimated with the same length-$R$
rolling window as $\yh_{t+1}$. Now the estimated linear model's prediction errors,
$\eph_{t+1}$, replace $\ep_{t+1}$ in the \oos\
test statistic. Unfortunately, $\eph_{t+1}$ is not an
\mds\ even when $\ep_{t+1}$ is, so the statistic is no longer
asymptotically mean-zero normal, even though this approximation
performs well in simulations. Since the window length is finite,
the estimator of $\btrue$ does not converge to $\btrue$.

This paper proposes using the same basic \oos\ statistic,
but using a recursive window to estimate $\btrue$ and produce
$\eph_{t+1}$:
\begin{align}
  \label{eq:8}
  \bh_t &= \Big(\sum_{s=1}^{t-1} x_{s} x_{s}'\Big)^{-1}
  \sum_{s=1}^{t-1} x_{s} y_{s+1}
  && \text{and}
  &
  \eph_{t+1} &= y_{t+1} - x_t'\bh_t
\end{align}
for each $t$.%
\footnote{The matrix inversion in $\bh_t$ can be replaced with a
  pseudo-inverse if necessary for some values of $t$ without changing
  the forecast.} %
\citepos{Wes:96} Theorem 4.1 implies that
\begin{equation*}
  \oclt{t} \Big[(y_{t+1} -
  x_t\bh_t)^2 - (y_{t+1} - v_t)^2 + (x_t'\bh_t - v_t)^2 \Big]
\end{equation*}
is asymptotically normal with mean zero under Clark and West's
\mds\ null for fairly
arbitrary processes $v_t$, as long as $v_t$ is weakly dependent and
the \oos\ statistic has uniformly positive variance.  Just as in
\citet{ClW:06,ClW:07}, these conditions are ensured if $v_t$ is
another forecast of $y_{t+1}$ based on a fixed-length rolling window.

So far, we have presented an especially simple version of the result
to make the intuition as clear as possible. The next section lists the
specific assumptions for the more general case and defines additional notation.

\subsection{Theoretical assumptions}
\label{sec:1a}

Consider the following environment. There is a single linear
benchmark model of the target variable, $y_{t+1}$:
\begin{equation}\label{eq:1}
  y_{t+1} = x_t'\beta + \ep_{t+1}, \quad t = 1,\dots,T-1
\end{equation}
where $\beta$ is an unknown vector of parameters and $x_t$ is an
observed vector of predictors. The parameter $\beta$ is estimated with
\ols\ using a recursive window as described by Equation~\eqref{eq:8}.
The alternative model is denoted $\yh_{t+1}$ and is estimated with a
rolling window of length $R$.

The main conditions on the \dgp\ are summarized in the first
assumption.  The weak dependence and moment conditions are
standard. The assumption of strict stationarity is stronger than
necessary in practice --- once the alternative forecasting method is
known, it is only necessary that the \oos\ adjusted loss difference be
weak stationary, and even that can be relaxed further --- but this
stronger assumption ensures that the results hold generally.

\phantomsection
\addcontentsline{toc}{subsubsection}{Assumption \ref{a1}}
\begin{asmp}\label{a1}%
  The data are generated by the relationship
  \begin{equation}
    y_{t+1} = x_t'\btrue + \ep_{t+1}
  \end{equation}
  for $t=1,2,\dots$, for some value $\btrue$, with $\E x_t \ep_{t+1} =
  0$, $\E \ep_{t+1}^2 > 0$, and $\E x_t x_t'$ positive definite for
  all $t$. Also assume that there is an additional sequence of random
  vectors $z_t$ and the process $(\ep_{t+1}, x_t, z_t)$ is stationary
  and strong mixing of size $-r/(r-2)$ or uniform mixing of size
  $-r/(2r-2)$, for $r > 2$.
\end{asmp}

The next assumption defines the forecasting models and adds additional
constraints to the \dgp.

\phantomsection
\addcontentsline{toc}{subsubsection}{Assumption \ref{a3}}
\begin{asmp}\label{a3}%
  The benchmark forecast is $x_t'\bh_t$, where $\bh_t$ is constructed
  with a recursive window according to~\eqref{eq:8}. The alternative
  forecast satisfies
  \begin{equation}
    \yh_{t+1} = \psi(y_t,z_t,\dots,y_{t-R+1}, z_{t-R+1})
  \end{equation}
  where $\psi$ is a known measurable function and the window length,
  $R$, remains finite as $T \to \infty$. Moreover, the vector
  $(\ep_{t+1}, x_t, \yh_{t+1})$ has uniformly bounded $2 r$ moments
  where $r$ is first defined in Assumption~\ref{a1}.
\end{asmp}

The requirement that the alternative forecast satisfies moment
conditions, rather than the underlying predictors $z_t$, is somewhat
unappealing but necessary. The function $\psi$ that generates these
forecasts is otherwise nearly unrestricted, so even well-behaved predictors
could produce arbitrarily badly-behaved forecasts. For example, if
\begin{equation*}
  z_t \sim \iid~\bernoulli(1/2),
\end{equation*}
setting $\psi(y_t, z_t) = 1/z_t$ would prevent a \clt\ from holding
since the forecast equals positive infinity with probability $1/2$. It
is easy to construct less obvious examples of problematic functions as
well. Assumption~\ref{a3} implicitly rules out these functional forms
by imposing moment conditions on the alternative models' forecasts.

Our next assumption ensures that the asymptotic variance of the \oos\
average is positive.
\phantomsection
\addcontentsline{toc}{subsubsection}{Assumption \ref{a4}}
\begin{asmp}\label{a4}%
  The asymptotic variance-covariance matrix
  \begin{equation}
    \var \Bigg(
      \oclt{t} \begin{pmatrix} x_t \\ \yh_{t+1} \end{pmatrix} \ep_{t+1}
      \Bigg)
  \end{equation}
  is uniformly positive definite (in $T$).
\end{asmp}
This assumption is much less restrictive than in \cite{Wes:96}.  As in
\cite{GiW:06} and \citet{ClW:06,ClW:07}, the assumption only serves to
rule out pathological cases --- for example, letting the alternative
model consist of only the first regressor of the benchmark. In \citet{Wes:96}, this
assumption is a restriction on the \dgp\ as well as the forecasting
models, but in this paper it is a restriction only on the models.

The final assumption restricts the class of \hac\ variance estimators
we will consider. We use the same class of estimators studied by
\citet{JoD:00} (their class $\mathcal{K}$); see their paper for
further discussion.
\phantomsection
\addcontentsline{toc}{subsubsection}{Assumption \ref{a5}}
\begin{asmp}\label{a5}%
  The kernel $K$ is a function from $\Re$ to $[-1,1]$ such that $K(0) = 1$, $K(x)
  = K(-x)$ for all $x$, $K(\cdot)$ is continuous at zero and all but a
  finite number of points, and
  \begin{gather*}
    \int_{-\infty}^{\infty} \lvert K(x) \rvert\, dx < \infty,
    \intertext{and}
    \int_{-\infty}^{\infty} \Bigg\lvert
    \int_{-\infty}^{\infty} K(z) e^{ixz}\,dz \Bigg\rvert\, dx < \infty.
  \end{gather*}
\end{asmp}

Last, we define some notation that will be used to derive the
theoretical properties of our \oos\ statistics.  The information set
that contains the information available for forecasting $y_{t+1}$ is
\begin{equation*}
  \Fs_t = \sigma(y_t, x_t, z_t, y_{t-1}, x_{t-1}, z_{t-1},\dots).
\end{equation*}
The adjusted \oos\ loss difference using a hypothetical value of
$\beta$ to produce the benchmark forecast is denoted by
\begin{equation*}
  f_t(\beta) = (y_{t+1} - x_t'\beta)^2 - (y_{t+1} - \yh_{t+1})^2 + (x_t'\beta - \yh_{t+1})^2.
\end{equation*}
Define the additional terms $\fh_t = f_t(\bh_t)$, $f_t = f_t(\btrue)$,
\begin{gather*}
  \gh_t = 2 \Bigg[\oavg{s} (x_s'\bh_s - \yh_{s+1}) x_s'\Bigg]\,
          \Bigg[\tfrac{1}{T-1} \sum_{s=1}^{T-1} x_s x_s'\Bigg]^{-1} x_t \eph_{t+1}
  \intertext{and}
  g_t = 2 \E\Big[(x_t'\btrue - \yh_{t+1}) x_t'\Big] \, (\E x_t x_t')^{-1} x_t \ep_{t+1}
\end{gather*}
and the \oos\ averages $\fb = \osum{t} \fh_t/P$, $\fb^* = \osum{t}
f_t/P$, $\gb = \osum{t} \gh_t/P$, and $\gb^* = \osum{t} g_t/P$.

\subsection{Theoretical results}
\label{sec:1b}

Asymptotic normality of the \oos\ average now follows directly from the
first three assumptions without other conditions. The proof is
presented in the Appendix and follows \citet{Wes:96} closely.

\phantomsection
\addcontentsline{toc}{subsubsection}{Theorem \ref{res:1}}
\begin{thm}\label{res:1}If Assumptions~\ref{a1} -- \ref{a4} hold then
$\sqrt{P} (\fb - \E \fb(\btrue)) \to^d N(0, \Sigma)$,
with
\begin{align*}
    \Sigma &= S_{ff} + S_{fg} + S_{fg}' + 2 \, S_{gg}, &
    S_{ff}  &= \lim P \var(\fb(\btrue)), \\
    S_{fg}  &= \lim P \cov(\fb(\btrue), \gb(\btrue)), &
    S_{gg}  &= \lim P \var(\gb(\btrue)).
\end{align*}
\end{thm}

To use this result, we need a consistent estimator of
$\sigma^2$. Define the \hac\ covariance estimator $\sigmah^2_1 =
\sh_{11} + 2 (\sh_{12} + \sh_{13})$ and the \mds\ covariance estimator
$\sigmah^2_2 = \sh_{21} + 2(\sh_{22} + \sh_{23})$ with
\begin{align*}
  \sh_{11} &= \oavg{s,t} (\fh_s - \fb) (\fh_t - \fb) K(\tfrac{t-s}{P}), &
  \sh_{21} &= \oavg{t} (\fh_t - \fb)^2, \\
  \sh_{12} &= \oavg{s,t} (\fh_s - \fb)(\gh_t - \gb) K(\tfrac{t-s}{P}), &
  \sh_{22} &= \oavg{t} (\fh_t - \fb)(\gh_t - \gb),
\intertext{and}
  \sh_{13} &= \oavg{s,t} (\gh_s - \gb) (\gh_t - \gb), &
  \sh_{23} &= \oavg{t} (\gh_t - \gb)^2.
\end{align*}

These estimators are consistent under similar assumptions to
Theorem~\ref{res:1}.

\phantomsection
\addcontentsline{toc}{subsubsection}{Lemma \ref{lem:2}}
\begin{lem}\label{lem:2}If Assumptions~\ref{a1}~--~\ref{a5} hold then
\begin{equation*}
  \sigmah_1^2 \to^p \sigma^2.
\end{equation*}
If Assumptions~\ref{a1}~--~\ref{a4} hold and $\{\varepsilon_{t},
\Fs_t\}$ is an \mds\ then
\begin{equation*}
   \sigmah_2^2 \to^p \sigma^2.
\end{equation*}
\end{lem}

Note that these results allow misspecification; asymptotic normality
follows from the weak dependence of the underlying series and from the
design of the test statistic. These statistics have typically been
used to test the null hypothesis that the benchmark model is correctly
specified --- that $\{\ep_t, \Fs_t\}$ is an \mds\ --- which
implies that $f_t$ is an \mds\ as discussed
at the beginning of this section. This is especially appealing in our
framework, since the benchmark can be theoretically motivated so the
\mds\ null would be a test of rationality. For example, \citet{GoW:08}
test whether excess returns for the S\&P 500 are predictable
out-of-sample, and any deviation of $\ep_{t+1}$ from an \mds\ is potentially
interesting. But the \mds\ null hypothesis only affects the estimator
of $\sigma^2$ (see Lemma~\ref{lem:2}); Theorem~\ref{res:1} continues
to hold under any \dgp\ that satisfies Assumptions~\ref{a1}~--~\ref{a4}.

In other settings, a researcher may want to test the weaker hypothesis
that $\E \fb^* = 0$ but the benchmark may be misspecified. Our
statistic can then be interpreted as an encompassing test as in
\citet{HLN:98}, and would test whether the alternative model contains
additional information that could make the benchmark model more
accurate. This interpretation can be motivated by the combination forecasting
model
\begin{equation*}
  \yh_{\mathit{avg},t+1} = (1 - w) x_t'\btrue + w \yh_{t+1}
\end{equation*}
which can be rewritten in terms of forecast errors as
\[
y_{t+1} - \yh_{\mathit{avg},t+1} = \ep_{t+1} + w (x_t'\btrue - \yh_{t+1}).
\]
The value
\[
w = \frac{\E \ep_{t+1} (\yh_{t+1} - x_t'\btrue)}{\E (x_t'\btrue - \yh_{t+1})^2}
\]
minimizes the \mse\ of the combination forecast, so the combination
model will have smaller \mse\ than the benchmark model, implying that
the alternative uses information not in the benchmark, unless
$\ep_{t+1}$ and $\yh_{t+1} - x_t'\btrue$ are uncorrelated. This
correlation is exactly the quantity measured by our statistic.

The final result puts together Theorem~\ref{res:1} and
Lemma~\ref{lem:2} to produce our test statistics. The null hypothesis
under misspecification is written in terms of $\E \ep_{t+1} \yh_{t+1}$
and not $\E \ep_{t+1} (\yh_{t+1} - x_t'\btrue)$, since $\E \ep_{t+1}
x_t = 0$ by construction. This result is an immediate consequence of
the previous two results and its proof is omitted.

\phantomsection
\addcontentsline{toc}{subsubsection}{Theorem \ref{thm:3}}
\begin{thm}\label{thm:3}If Assumptions~\ref{a1}--\ref{a5} hold, then
\begin{equation*}
  \sqrt{P} \fb / \sigmah_1 \to^d N(0, 1)
\end{equation*}
under the null hypothesis  $\E(\varepsilon_{t+1} \yh_{t+1}) = 0$ for all $t = R,\dots,T-1$.
If, instead, Assumptions~\ref{a1}--\ref{a4} hold, then
\begin{equation*}
  \sqrt{P} \fb / \sigmah_2 \to^d N(0, 1)
\end{equation*}
under the null hypothesis that $\{\varepsilon_t, \Fs_t\}$ is an \mds.
\end{thm}

The test statistic proposed in Theorem~\ref{thm:3} can be easily
extended in several ways. For longer-horizon forecasts (two or more
periods ahead), $\sigmah_1$ will remain consistent but $\sigmah_2$
will not --- the forecast errors for a correctly specified
$h$-step-ahead forecast have an MA($h-1$) dependence structure --- but
using a generalized $\sigmah_2$ that reflects this covariance
structure restores consistency. To test optimality under loss
functions other than squared-error, one can replace the forecast error
with the generalized forecast error \citep[see, for
example][]{PaT:07,PaT:07b} and replace the \ols\ estimator of $\beta$ with the
corresponding $M$-estimator. And the benchmark model can be replaced
in general with a nonlinear model that satisfies the assumptions of
\citet{Wes:96} or \citet{Mcc:00} by making the appropriate changes to
$f_t$ and $g_t$. (See \citealp{Wes:96}, and \citealp{Mcc:00},
for details.) The general approach of using a recursive window to
estimate the benchmark and a fixed-length rolling window to estimate
the alternative applies quite broadly.

%%% Local Variables:
%%% mode: latex
%%% TeX-master: "mixedwindow"
%%% TeX-command-extra-options: "-shell-escape"
%%% End:
